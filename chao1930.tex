\documentclass{article}

\setlength{\parindent}{0em}
\setlength{\parskip}{1.1em}

\usepackage{geometry}

\usepackage[tone]{tipa}
\newcommand{\T}[1]{\textipa{\tone{#1}}}
\newcommand{\I}[1]{\textipa{#1}}

\usepackage{framed}
\usepackage{url}

% The package "tipa" puts the reference bar of tone letters on the *right*.
% However, we need \reflectbox{} from the "graphicx" package
% for tone letters with reference bar on the *left*,
% for English intonation on page 26 of Chao (1930).
% Example: for intonation pattern that would be \T{2431},
% use \reflectionbox{\T{1342}} instead
% (using reflectionbox and reversing the left-to-right order of the numbers)

% TODO: implement a command like \TRightBar{} that basically does what \T{}
% (defined above) does but with the reference bar on the left and without
% the mental gymnastics of having to reverse the ordering of numbers.

\usepackage{graphicx}

\begin{document}

\begin{framed}

This is the Chao 1930 paper `A system of ``tone-letters''' typeset in ordinary English orthography by Jackson Lee. For its \LaTeX\ source code as well as a scan of the original paper written in IPA, please see this repository: \\
\url{https://github.com/JacksonLLee/chao1930}

\end{framed}

\begin{center}

A system of ``tone-letters''

Yuen Ren Chao

1930

{\em Le Ma\^itre Phon\'etique} 45, 24-27

\end{center}


With a view to combine accuracy, elegance, and convenience for printing, I have devised the following system of ``tone-letters'' for the consideration of fellow phoneticians.

Each tone-letter consists of a vertical reference line, of the height of an N, to which a simplified time-pitch curve of the tone represented is [page 25 begins] attached, for tonemes to the left of the line, and for tone-values to its right. The thickness of the lines is to be the same as the horizontal (thin) elements of the Roman character.

The total range is divided into four equal parts, thus making five points, numbered 1, 2, 3, 4, 5, corresponding to low, half-low, medium, half-high, high, respectively. In order not to make distinctions too fine, points 2 and 4 are used either alone or with each other, but not in combination with 1, 3, or 5. With this restriction, the total number of tone-letters which I propose is as follows:

\begin{tabular}{cccccc}
\multicolumn{2}{c}{straight tones} & \multicolumn{2}{c}{circumflex tones} & \multicolumn{2}{c}{short tones} \\
tone-letter & name & tone-letter & name & tone-letter & name \\
\T{111} & 11: & \T{131} & 131: & \T{1} & 1: \\
\T{13}  & 13: & \T{153} & 153: & \T{2} & 2: \\
\T{15}  & 15: & \T{242} & 242: & \T{3} & 3: \\
\T{222} & 22: & \T{313} & 313: & \T{4} & 4: \\
\T{24}  & 24: & \T{315} & 315: & \T{5} & 5: \\
\T{31}  & 31: & \T{351} & 351: \\
\T{333} & 33: & \T{353} & 353: \\
\T{35}  & 35: & \T{424} & 424: \\
\T{42}  & 42: & \T{513} & 513: \\
\T{444} & 44: & \T{535} & 535: \\
\T{51}  & 51: \\
\T{53}  & 53: \\
\T{555} & 55: \\
\end{tabular}

These are the toneme signs. The same ``curves'' with the vertical line placed on the right side will give the signs for tone-values.

As the intervals of speech-tones are only relative intervals, the range 1--5 is taken to represent only ordinary range of speech intonation, to include cases of moderate variation for logical expression, but not to include cases of extreme emotional expression. For purposes of tone drills, each step may be taken to be a whole tone, thus making the total range equal to an augmented fifth. This would make the successive pronunciation of a number of tones sound rather unmusical, which however is rather an advantage for phonetic purposes.


It will be noted that no double circumflex tones are given in the list. When such a tone is spread over more than one syllable, it [page 26 begins] can be separated into components which are in the list. Where such a tone applies to one syllable, the sign may be made up as occasions arise; thus Palmer's third tone for English may be given as \reflectbox{\T{3153}} (:3513).

Examples of use:

\begin{tabular}{lll}
:42 & \I{jes}\reflectbox{\T{24}} & (ordinary affirmation.) \\
:51 & \I{jes}\reflectbox{\T{15}} & (of course.) \\
:24 & \I{jes}\reflectbox{\T{42}} & (go on, I'm anxious to hear the rest of it.) \\
:13 & \I{jes}\reflectbox{\T{31}} & (I'm listening.) \\
:15 & \I{jes}\reflectbox{\T{51}} & (but, ---.) \\
:11 & \I{Hjes}\reflectbox{\T{11}} & (I understand of course.) \\
:44 & \I{j\~e's}\reflectbox{\T{44}} & (It's alright, although you made a mess of it.) \\
:55 & \I{j\~e's}\reflectbox{\T{55}} & (I heard all about that sort of thing.) \\
:351 & \I{jes}\reflectbox{\T{153}}  & (I should be most delighted.) \\
:3513 & \I{jes}\reflectbox{\T{3153}} & (so far as that's concerned, only ---.) \\
\end{tabular}

\begin{tabular}{lllll}
\I{\*wE@}\reflectbox{\T{55}} & \I{d@z}\reflectbox{\T{44}} & \I{i:}\reflectbox{\T{33}} & \I{liv}\reflectbox{\T{13}} & (Ordinary interrogation.) \\
\I{\*wE@}\reflectbox{\T{11}} & \I{d@z}\reflectbox{\T{33}} & \I{i:}\reflectbox{\T{44}} & \I{liv}\reflectbox{\T{55}} & (Where did you say he lived?) \\
\I{\*wE@}\reflectbox{\T{11}} & \I{d@z}\reflectbox{\T{22}} & \I{i:}\reflectbox{\T{44}} & \I{liv}\reflectbox{\T{15}} & (No matter where he eats.) \\
\I{\*wE@}\reflectbox{\T{53}} & \I{d@z}\reflectbox{\T{33}} & \I{i:}\reflectbox{\T{11}} & \I{liv}\reflectbox{\T{53}} & (I didn't ask..., I asked {\em how} he lived.) \\
\I{\*wE@}\reflectbox{\T{11}} & \I{d@z}\reflectbox{\T{44}} & \I{i:}\reflectbox{\T{55}} & \I{liv}\reflectbox{\T{51}} & (Don't you know where he lives?) \\
\end{tabular}

Cantonese:

\I{NO\tone{13} i\tone{11}kA\tone{53}\tone{55} wA\tone{22} pei\tone{25} nei\tone{13} tCi\tone{53}, kOk\tone{33} jAm\tone{53} kE\tone{33} CAn\tone{53}mEN\tone{11}\tone{35}, tChi\tone{11}hA\tone{13} tC\oe N\tone{53} kOk\tone{33} jAm\tone{53} kE\tone{33} kA:u\tone{22}MEN\tone{11}\tone{35}, thuN\tone{11} mA:i\tone{11} tChi\tone{33}tC\oe y\tone{22}, kON\tone{35} pei\tone{35} nei\tone{13} thEN\tone{53}. kO\tone{35} luk\tone{22}kO\tone{33} jAm\tone{53} tCAu\tone{22} hAi\tone{22} ni\tone{53}\tone{55} luk\tone{22}kO\tone{33} tCi\tone{22} kE\tone{33} jAm\tone{53}:}

\begin{centering}

\I{fAn}\tone{53} \I{fAn}\tone{35} \I{fAn}\tone{33} \I{fAn}\tone{11} \I{fAn}\tone{13} \I{fAn}\tone{11}\footnote{See Daniel Jones and Kwing Tong Woo, {\em A Cantonese Phonetic Reader}, p. 17.}.

\end{centering}

\newpage

Tibetan (Lhasa dialect)

(1) Transliteration. (A voiceless initial goes with the high tone \T{53} and a voiced initial goes with the low tone \T{131}, unless marked to the contrary.) \\

\newcommand{\Indent}{
\setlength{\parindent}{3em}
\setlength{\parskip}{0em}
}

\Indent


\I{\*rAN lA kA\tone{131} wE tChAm\tone{131} pA}

\I{Cen\tone{131} tCi\tone{131} tyn\tone{131} mA\*r lAN\tone{53} soN}

\I{khoknE sempE tCoN khi\tone{131}}




\setlength{\parindent}{0em}
\setlength{\parskip}{1em}

[page 27 begins] \\

\Indent

\I{lyp\o\ CA jAN kAm\tone{131} soN}

\I{\textltailn in\tone{53} thup ku\tone{131} lA Co\*r soN}

\I{motChA tsipyl\tone{131} \*ren soN}

\I{phu\tone{131} mo thuN\tone{131} semtCenmA}

\I{mi\tone{53} lAm lA kho\*r soN}.


\setlength{\parindent}{0em}
\setlength{\parskip}{1em}

(2) The same as actually pronounced: \\


\Indent


\I{\*rAN}\reflectbox{\T{31}} \I{lA}\reflectbox{\T{11}} \I{kA}\reflectbox{\T{53}} \I{wE}\reflectbox{\T{22}} \I{tChAm}\reflectbox{\T{11}} \I{bA}\reflectbox{\T{33}}

\I{Cen}\reflectbox{\T{53}} \I{d\textctz I}\reflectbox{\T{22}} \I{tym}\reflectbox{\T{11}} \I{m@r}\reflectbox{\T{33}} \I{lAN}\reflectbox{\T{35}} \I{soN}\reflectbox{\T{11}}

\I{kho\super k}\reflectbox{\T{44}}\I{nE}\reflectbox{\T{44}} \I{sem}\reflectbox{\T{55}}\I{bE}\reflectbox{\T{55}} \I{tCoN}\reflectbox{\T{35}} \I{khi}\reflectbox{\T{11}}

\I{l0}\reflectbox{\T{33}}\I{b\o}\reflectbox{\T{53}} \I{CA}\reflectbox{\T{44}} \I{jAN}\reflectbox{\T{22}} \I{kAm}\reflectbox{\T{35}} \I{soN}\reflectbox{\T{11}}

\I{\textltailn in}\reflectbox{\T{44}}\I{tup}\reflectbox{\T{44}} \I{ku}\reflectbox{\T{44}} \I{lA}\reflectbox{\T{22}} \I{Co\*r}\reflectbox{\T{22}} \I{soN}\reflectbox{\T{11}}

\I{m\~o}\reflectbox{\T{22}}\I{tC@}\reflectbox{\T{55}} \I{tsI}\reflectbox{\T{44}}\I{b0l}\reflectbox{\T{44}} \I{\*ren}\reflectbox{\T{53}} \I{soN}\reflectbox{\T{11}}

\I{phu}\reflectbox{\T{33}}\I{m\~o}\reflectbox{\T{44}} \I{thuN}\reflectbox{\T{11}}\I{sIm}\reflectbox{\T{35}}\I{tCem}\reflectbox{\T{33}}\I{mA}\reflectbox{\T{44}}

\I{mi}\reflectbox{\T{55}}\I{lAm}\reflectbox{\T{55}} \I{lA}\reflectbox{\T{22}} \I{khor}\reflectbox{\T{44}} \I{soN}\reflectbox{\T{11}}.\footnote{{\em Sixty-two Tibetan Folk-songs of Tsha\.{n}-dbya\.{n}s-rgya-tsho}, translated [into Chinese] by Yu Dawchyuan and transcribed by Jaw Yuanrenn, to be published soon by The Institute of History and Philology of Academia Sinica.}


\setlength{\parindent}{0em}
\setlength{\parskip}{1em}




The practical value of any system of notation depends on the possibility of its working both ways. As a test for this requirement, I used the system in recording sixty-two Tibetan folk-songs spoken (not sung) to a dictaphone, from which the transcription was made. Then, after leaving the thing alone for several days, I picked up my manuscript again and read the whole transcription, tone and all, back into dictaphone records. On playing the original and my fake Tibetan pronunciation on two machines, and comparing them sentence by sentence in close succession, the resemblance between the two was beyond my expectation. This clearly shows that it is possible to train oneself in such a system so as to make it work backwards as well as forwards.



\end{document}
